\documentclass[a4paper]{scrartcl}

\usepackage[english]{babel}
\usepackage[utf8]{inputenc}
\usepackage[T1]{fontenc}

\title{Template File}

\begin{document}

\maketitle

Hello #(latex .REPLNAME#),

This is just a file demonstrating you how to use the gummibaum template system.
Here is some verbatim text: #(verb "|" "foo" "bar" 42#).

Here is another constant #(latex .REPLFOO#)

You can also use it to iterate over data from a csv file.

% The minus suppresses new line at the end of the template instruction
#(range .data.Columns -#)
	Token is #(latex .Map.token#) and value is #(latex .Map.value#)\\
#(end#)

You can also iterate over each entry in head.

\begin{tabular}{ll}
#(if .data.Head#)
	#(- join " & " .data.Head#)\\
#(end -#)
\end{tabular}

Here's another example using the join function #(join "," "foo" "bar"#)

Here's an example to get an entry by index, get first column and get entry on position 0:

#(with $col := (index .data.Columns 0) -#)
	#(- latex (index $col.Entries 0)#)
#( end -#)
\end{document}
